\subsubsection{Modelagem da superfície}\label{modelagem}

Existem diversas abordagens matemáticas para descrição de superfícies como:
Parametrização Polinomial; Polinômios em três variáveis; Superfícies de
Bézier (\cite{farin2002curves}); Splines e NURBS (\textit{Non-uniform
rational B-spline}); Subdivisão de superfícies (\cite{peters2008subdivision}); Malhas
poligonais.

Todas essas formas de representar uma superfície, com excessão das malhas,
recaem em alguma instância em uma descrição polinomial. Dentre essas a única
descrição que ocorre de maneira implícita é por polinômios em três variáveis,
descrevendo uma variedade algébrica bidimensional, enquanto as demais são
parametrizações da superfície. 

Por simplicidade, fácil manipulação algébrica e implementação, a descrição
puramente polinomial (implícita) foi escolhida como abordagem inicial. De
maneira geral a superfície é descrita como o conjunto solução sobre os
números reais da equação polinomial ($f(x,y,z)=0$) de grau $N$, dito grau da
superfície, em $x$,$y$ e $z$:
\[\sum\limits_{i+j+k \leq N}^{} C_{i,j,k}x^iy^jz^k = 0\]

Os coeficientes $C_{i,j,k}$, então, são aqueles que descrevem da superfície.
Devido a restrição do grau do polinômio, o número de coeficientes é
$\binom{N+3}{3}$. Podendo ser vistos como coordenadas da superfície num espaço
projetivo de dimensão igual ao número de coeficiente,
$\mathbb{P}^{\binom{N+3}{3}}$, em outras palavras a superfície é invariante a
escalamento dos coeficientes.

Experimentalmente foi indentificado que um polinômio de quarto grau é suficiente
para aproximar toda uma região de interesse da pá, onde será feito o revestimento
para uma posição do robô, com erro submilimétrico. Nesse caso, o número de
coeficientes que devem ser identificados é $\binom{7}{3}$, ou seja, 35.

Com base no artigo de \cite{juttler2002least}, a conversão da descrição da
superfície de nuvem de pontos para uma descrição analítica foi feita utilizando
a informação da direção da normal à superfície em cada ponto, ou seja, a superfície analítica deve não apenas passar
próxima aos pontos da nuvem como deve também ter seu vetor normal similiar à
normal desses pontos.

Explorando o fato que polinômio são lineares em seus coeficientes, um sistema
superdeterminado, a ser resolvido por mínimo quadrados(\textit{curve fitting}
\cite{arlinghaus1994practical}) , foi contruído a partir do cálculo dos termos
do polinômio em cada ponto da nuvem (fazendo $f(x,y,z)=0$) e da avaliação da
normal em cada ponto, que deveria concordar com o gradiente do polinômio (ou
seja $\nabla f(x,y,z) = \overrightarrow{n}$, onde $\overrightarrow{n}$ é a
normal no ponto $(x,y,z)$ da nuvem), dessa forma o peso dado aos vetores normais
das amostras é igual ao peso das amostras.

%Elael Modelo Polinomial multivarial -> extrapolado para a pá inteira
% PREMISSAS!!


% 
% Splines
% Bézier Surface
% Runge's phenomenon
% Multivariate Polynomial fitting
