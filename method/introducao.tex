\section{Introdução}
%TODO  Gabriel: contextualização do problema, sota, Jirau

A metodoligia empregada durante o desenvolvimento do projeto EMMA consistiu em
diversas etapas, que alimentavam a seguinte e, caso necessário,
realimentavam uma etapa anterior para refinamento da solução ou alinhamento de
resultados. 

Primeiramente, em \ref{cap::sota} foi realizada uma pesquisa e delimitação
do escopo do problema, etapa fundamental para o completo entendimento do
problema e responsável por direcionar o esboço das primeiras soluções. A
pesquisa sobre o estado da arte mostrou que nenhuma solução disponível no
mercado e no meio acadêmico é capaz de suprir completamente as necessidades
presentes no processo de metalização \textit{in situ}, destacando as limitações
de acesso e espaço confinado como os maiores desafios técnicos e logísticos a
serem enfrentados. A fim de tentar construir uma solução mais geral, foi
determinada a utilização da escotilha inferior como acesso principal e a soluçao
conceitual inicial consiste em um manipulador robótico sobre trilhos modulares.

Como visto em \ref{cap::detail},  foi realizada uma pesquisa de mercado listando
os manipuladores comerciais disponíveis e que satisfaziam as restrições de alcance
e peso. Entretanto a configuração de cada manipulador, ou seja, a disposição
de seus elos e juntas, influencia em seu espaço de trabalho e um estudo sobre o
espaço de trabalho e a cinemática do manipulador foi necessária. O
manipulador Motoman\textregistered MH12 foi eleito para integrar a solução, pois
possui a capacidade de recobrimento de quase todo o alcance vertical da pá
(revestimento de cima a baixo), capacidade de carga suficiente para a operação e peso dentro das
restrições impostas. Paralelamente, a estratégia de posicionamento do sistema e
a calibração de seus componentes foi verificada, determinando os sensores a
serem utilizados, afim de satisfazer as restrições de precisão, robustez e
segurança.

Neste documento será confirmada a real viabilidade de
sua utilização, verificando que é possível a total cobertura da pá durante o
processo de revestimento. Análises cinemática, dinâmica serão realizadas com
o auxílio de ferramentas de simulação como a plataforma OpenRAVE. Em seguida a
estratégia de controle e planejamento de trajetória tem o objetivo de assegurar
que a movimentação do manipulador se desenvolva de forma contínua em todo o
espaço de juntas e que as velocidades e acelerações máximas sejam respeitadas.

Por sua vez, a análise detalhada de cobertura da pá, fornece os requisitos
mínimos que a base mecânica deve obedecer, como forças exercidas e graus de liberdade necessários
para alcancar todas as posições da base do manipulador. A partir dos conceitos
analisados em \ref{cap::detail}, pode-se comparar diferentes soluções e as
vantagens de desvantagens de cada uma. Foi escolhido o conceito
Prismático-Rotacional-Prismático-Prismático (P-R-P-P) porque mostrou-se a
solução mais viável construtivamente. Foi realizado então o projeto básico da
solução e o dimensionamento dos componentes. Os resultados do dimensionamento
permitirão o detalhamento final da base mecânica, compra de materiais, montagem
e testes.

A operação de montagem do trilho e posicionamento do robô, assim como
a rotação do rotor, isto é, o ângulo em que a pá se encontra, não são fixos ou
precisamente reprodutíveis e por isso a calibração deve ser realizada e as
transformadas de coordenadas entre o manipulador e a pá devem ser econtradas.
Para localizar o manipulador é estudada a utilização de marcadores e o
posicionamento da pá será encontrado por meio da análise dos dados
tridimensionais provenientes do sensor Faro\textregistered Focus X330