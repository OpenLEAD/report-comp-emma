\subsection{Reaplicação de revestimento sobre superfície desgastada}
Com o intuito de testar a metalização sobre o revestimento já desgastado da pá
e o seu desempenho quanto à adesão, um teste de aplicação sobre revestimento
desgastado foi conduzido. Foi utilizada a seguinte metodologia: jateamento dos
corpos de prova metálicos; aplicação de 0.1 mm de camada de WC; medição da
rugosidade; polimento do revestimento com lixa 200; medição da rugosidade; jateamento
sobre superfície revestida e polida; medição de rugosidade; metalização sobre
revestimento polido e jateado; execução dos ensaios de adesão e dobramento e
metalografia.

Para avaliação do desempenho de aspersão térmica aplicada sobre revestimento
remanescente de uma peça, foi necessário realizar um teste de revestimento com
desgaste, jateamento sobre o revestimento, e a avaliação da adesão do conjunto.
Após as medições, foi realizado um polimento da superfície com lixa 300 para
desgastar o revestimento de carboneto de Tungstênio. No momento em que a
superfície ficou com desgaste na qual a rugosidade Ra ficasse abaixo de 1.0
micron, o polimento foi cessado e a medição de espessura foi realizada para
garantir que ainda havia camada de revestimento duro para posterior jateamento
e aspersão. 

Os testes de dobramento são realizados de forma comparativa com a
superfície somente jateado, sem revestimento prévio. No ensaio de dobramento,
avaliou-se que há formação de trincas na superfície do revestimento a partir
do dobramento em 90° de uma chapa com espessura de 4.0 mm em um mandril com
diâmetro de 16 mm. Comparativamente com a amostra sem revestimento prévio
poucas trincas apareceram e não houve destacamento do revestimento, com exceção
das bordas, porém em nível aceitável. Para o teste de adesão conforme a norma
ASTM C633 é considerado um resultado como garantia de adesão apropriada uma
tensão mínima de destacamento de 70 Mpa. O resultado do ensaio de adesão
atendeu a esse critério, com o rompimento se dando na cola utilizada para
realizar o ensaio.