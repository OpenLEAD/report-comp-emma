\subsection{Construção do ambiente de simulação}

Os componentes de interesse da turbina e o manipulador foram modelados no
software de projeto \textit{Computer Aided Desing} (CAD) 3D \textit{SolidWorks},
a partir dos desenhos técnicos fornecidos pela Energia Sustentável do Brasil (ESBR). Após a visita à unidade geradora, porém,
verificou-se que o modelo da pá da turbina continha inconsistências com a pá
real. Portanto foi realizado um mapeamento 3D pelo sensor Laser Scanner FARO
Focus3D. O sensor gera uma núvem de pontos, que é interpolada, gerando um
arquivo VRML, importado ao ambiente de simulação OpenRAVE.
