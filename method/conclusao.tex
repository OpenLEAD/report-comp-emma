\section{Conclusão}
%TODO TODOS: conclusoes finais

% -.-.-.-.-.-.-.-.-.-.-.-.-.-.-.-.-.-.-.-.-.-.-.-.-.-.-.-.-.-.-.-.-.-.-.-.-.-.-.
%Mecânica
As simulações pelo Método de Elementos Finitos verificaram a rigidez da
estrutura dada, a geometria imposta pelo conceito P-R-P-P e para as opções
comerciais disponíveis de material, perfil de alumínio estrutural, trilho e bases
magnéticas.
Os resultados se mostraram satisfatórios para os os componentes selecionados,
ressaltando que foram considerados casos extremos de operação. A flexibilidade
da estrutura causa erros com ordem de grandeza de $1~mm$, o que não interfere na
qualidade do processo.
As forças resultantes nos pontos de ancoragem permitem dimensionamento e seleção
das bases mecânicas para cada região de ancoragem, não limitando o mesmo tamanho
de base para todos os pontos.
Os resultados de integridade do componentes conferiram Fatores de Segurança
aceitáveis e dentro dos valores recomendados para projetos mecânicos em geral.
% -.-.-.-.-.-.-.-.-.-.-.-.-.-.-.-.-.-.-.-.-.-.-.-.-.-.-.-.-.-.-.-.-.-.-.-.-.-.-.


% -.-.-.-.-.-.-.-.-.-.-.-.-.-.-.-.-.-.-.-.-.-.-.-.-.-.-.-.-.-.-.-.-.-.-.-.-.-.-.
%Controle
No âmbito do controle e planejamento de trajetória, o método desenvolvido já se
mostrou capaz de analisar a superfície da pá, segmenta-la em regiões e definir
caminhos para serem percorridos, tanto pelo efetuador (no espaço de trabalho)
quanto pelas juntas (no espaço de juntas).

Porém, validações da precisão do resultado e análise de colisão devem ser mais
exploradas. Também devem ser julgadas pequenas modificações do método como
aplicação de minímos quadrados móveis na definição da superfície, técnica 
que possibilitaria um maior controle do erro localmente.

% -.-.-.-.-.-.-.-.-.-.-.-.-.-.-.-.-.-.-.-.-.-.-.-.-.-.-.-.-.-.-.-.-.-.-.-.-.-.-.
%Calibração

Por sua vez, a exploração das características trdimensionais do ambiente,
aquisitadas pelo sensor laser Faro Focus X330, se mostrou viável para a identificação da posição
e orientação do robô e da pá a ser processada.
A utilização de marcadores será empregrada sempre que possível. Entretanto para
a localicazação da pá, apenas a nuvem de pontos será utilizada, devido a
limitaçoes de instalção.
A simulação do ambiente da turbina possibilitou a implementação de um algoritmo preliminar e o
teste com a presença de oclusões e ruído.
