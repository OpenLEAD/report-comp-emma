\section{Metodologia do sistema de controle}
A metodologia para a construção de um sistema de controle para o manipulador do
projeto EMMA consiste nas seguintes etapas: 1) construção do ambiente de
simulação da área de interesse da turbina (pás, rotor, íris, acessos e aro
câmara), e modelagem do manipulador; 2) análises cinemática, dinâmica e
controle do manipulador; 3) planejamento de trajetórias e velocidades.

Nesta análise de viabilidade técnica, foi utilizada a 
multi-plataforma arquitetura de software de código aberto Open
Robotics and Animation Virtual Environment (OpenRAVE). OpenRAVE é voltado para
aplicações de robôs autônomos e possui simulação 3D, visualização, planejamento,
e controle. A arquitetura de plugin permite ao usuário desenvolver controladores
customizados ou estender funcionalidades. O desenvolvedor pode se
concentrar em planejamento e aspectos específicos do problema, sem necessitar
gerenciar explicitamente os detalhes de cinemática e dinâmica do robô,
detecção de colisão ou atualização do mundo (ambiente). Além disso, OpenRAVE
pode ser usado em conjunção com populares pacotes de robótica, como ROS, Player e MatLab
\cite{diankov2008openrave}. 

\subsection{Construção do ambiente de simulação}

Os componentes de interesse da turbina e o manipulador foram modelados no
software de projeto \textit{Computer Aided Desing} (CAD) 3D \textit{SolidWorks},
a partir dos desenhos técnicos fornecidos pela Energia Sustentável do Brasil (ESBR). Após a visita à unidade geradora, porém,
verificou-se que o modelo da pá da turbina continha inconsistências com a pá
real. Portanto foi realizado um mapeamento 3D pelo sensor Laser Scanner FARO
Focus3D. O sensor gera uma núvem de pontos, que é interpolada, gerando um
arquivo VRML, importado ao ambiente de simulação OpenRAVE.

\subsection{Análises cinemática, dinâmica e
controle do manipulador}

A análise da simulação revela as áreas da pá que podem ser revestidas,
as áreas de mais difícil acesso, e as posições da base do manipulador para a
execução bem sucedida da operação, levando em consideração a cinemática,
dinâmica e controle do manipulador. 

A metodologia percorre os seguintes tópicos: 1) discretização da pá; 2)
avaliação de revestimento dos extremos da pá, restringindo a base a um trilho
idealizado pelo projeto mecânico; 3) teste de revestimento completo; 4) teste de
revestimento de áreas até então não revestidas, sugerindo novas posições para a
base. Onde 1) já foi previamente avaliado em \textit{EMMA-DETAIL}. 

As variáveis do processo de revestimento são: 1) pistola de revestimento
é aproximada por um cilindro de 300 mm de comprimento e 50 mm de raio, 
acoplada à extremidade do efetuador; 2) as pás da turbina podem girar em seu
próprio eixo de $0^o$ a $29^o$; 3) o rotor da turbina pode girar de
$0^o$ a $360^o$; 4) a distância entre extremidade da pistola e pá deve
ser mantida em $235 \pm 5$ mm durante toda a operação; 4) o ângulo entre a
pistola e o plano da pá deve ser $90^o \pm 60^o$, mas é aconselhável uma
tolerância máxima de $30^o$.

Além disso, vale observar algumas conclusões feitas no relatório
\textit{EMMA-DETAIL}: 1) a análise cinemática verifica se, dado um ponto da
pá, há alguma configuração das juntas para revestí-lo; 2) a análise dinâmica
mostrou que, durante a execução de uma trajetória, se o manipulador estiver
muito próximo do ponto que deseja revestir, os ângulos das juntas tendem a
variar muito, aumentando drasticamente os torques e impossibilitando a operação.


\subsubsection{Avaliação dos extremos da pá} 

É fácil observar que as áreas de mais difícil acesso à pá são suas extremidades,
áreas em preto da figura~\ref{fig::extremidades}, devido ao alcance do robô e
por serem áreas em que a pá apresenta maior inclinação. A alteração da direção
do vetor normal à pá e a posição do ponto a ser revestido são os fatores 
importantes para a aplicação de revestimento, isto é, pontos em uma posição
elevada para o robô e vetor normal direcionada para cima são complexos. Observe
a figura~\ref{fig::normal}, o vetor $\vec{P}$ é um exemplo de vetor normal
pertencente a um ponto da pá de difícil revestimento, tanto pela posição quanto
pelo vetor normal.

\begin{figure}[!ht]
	\centering	
	\includegraphics[width=.8\columnwidth]{method/figs/tocoat.jpg}
	\caption{Extremidades da pá em preto.}
	\label{fig::extremidades}
\end{figure}

\begin{figure}[!ht]
	\centering	
	\includegraphics[width=.5\columnwidth]{method/figs/normal.png}
	\caption{Vetores normais (em preto) de pontos a serem revestidos (em
	vermelho).}
	\label{fig::normal}
\end{figure}

A simulação de teste de revestimento das extremidades considerou as seguintes
variáveis: 1) os ângulos das pás da turbina foram fixados em $24^o$, ângulo
natural de uso; 2) avaliou-se o giro do rotor de $0^o$ a $30^o$ com passo de
$3^o$; 3) a distância entre extremidade da pistola e pá pode variar $235 \pm 5$
mm; 4) o ângulo entre a pistola e o plano da pá pode variar $90^o \pm 30^o$; 5)
a posição em $y$ do robô pode variar $-2970 \pm 250$ mm (posição global) com
passo de 50 mm; 6) a posição em $x$ do robô pode variar $715 \pm 485$ mm em
relação à pá com passo de 50 mm; 7) a posição em $z$ do robô foi amostrada
uniformemente em 7 pontos ao longo da pá.

A figura~\ref{fig::trilho2all} mostra todas as posições simuladas para a base do
robô. As restrições dessas posições são a altura mínima da base ao aro câmara e
a altura do robô. Para cada posição de base e ângulo do rotor, foi simulado o
processo de revestimento para as quatro extremidades da pá, levando em
consideração as respectivas tolerâncias.

\begin{figure}[!ht]
	\centering	
	\includegraphics[width=.5\columnwidth]{method/figs/trilho2all.png}
	\caption{Possíveis posições da base do robô, em verde.}
	\label{fig::trilho2all}
\end{figure}

Nas subseções seguintes, serão analisadas cada extremidade da
pá independentemente, considerando as varíaveis acima.

\paragraph{Extremidade inferior esquerda}

A extremidade inferior esquerda apresenta a complexidade de posição do ponto a
ser revestido, visto que os pontos estão na borda, próximas ao aro câmara,
aumentando o risco de colisões. A figura~\ref{fig::footleft} mostra a
discretização da pá na extremidade esquerda: pontos azuis são pontos revestidos
na tolerância de $30^o$; em preto, pontos revestidos sem tolerância; e em
vermelho, pontos que não foram revestidos para esta posição do robô.

\begin{figure}[!ht]
	\centering	
	\includegraphics[width=.5\columnwidth]{method/figs/footleft.png}
	\caption{Estudo de revestimento para a extremidade inferior esquerda.}
	\label{fig::footleft}
\end{figure}

A simulação mostrou que o robô foi capaz de revestir toda a extremidade, na
altura mínima $y=-3220$ mm, a uma distância de até $x=1200$ mm da pá. Conforme o
robô se aproxima da pá, verifica-se que a altura pode sofrer variações
positivas, por exemplo para $x=980$ mm, $y=-3070$ mm. Entretanto, como já foi
verificado na simulação dinâmica, não é aconselhável aproximar o robô da pá a
uma distância inferior a $x=1000$ mm, pois os torques durante a execução podem ser
elevados.

Dessa forma, a extremidade inferior esquerda não mostrou desafios técnicos.

\paragraph{Extremidade inferior direita}

A extremidade inferior direita possui complexidade de posição do ponto a
ser revestido maior que a extremidade inferior esquerda, visto que a extremidade
está mais próxima da área em que a curvatura do aro é superior a $20^o$ (base
mais próxima do solo). A figura~\ref{fig::footright} mostra a discretização da
pá na extremidade esquerda: pontos azuis são pontos revestidos na tolerância de
$30^o$; em preto, pontos revestidos sem tolerância; e em vermelho, pontos que
não foram revestidos para esta posição do robô.

\begin{figure}[!ht]
	\centering	
	\includegraphics[width=.5\columnwidth]{method/figs/footright.png}
	\caption{Estudo de revestimento para a extremidade inferior direita.}
	\label{fig::footright}
\end{figure}

A simulação mostrou que o robô foi capaz de revestir toda a extremidade, na
altura mínima $y=-3220$ mm, a uma distância de até $x=1200$ mm da pá. 

Dessa forma, a extremidade inferior direita não mostrou desafios técnicos.

\paragraph{Extremidade superior esquerda}\label{superioresquerda}

A extremidade superior esquerda possui complexidade de posição do ponto a
ser revestido devido à altura do robô. A figura~\ref{fig::shoulderleft} mostra
a discretização da pá na extremidade esquerda: pontos azuis são pontos revestidos na tolerância de
$30^o$; em preto, pontos revestidos sem tolerância; e em vermelho, pontos que
não foram revestidos para esta posição do robô.

\begin{figure}[!ht]
	\centering	
	\includegraphics[width=.5\columnwidth]{method/figs/shoulderleft.png}
	\caption{Estudo de revestimento para a extremidade superior esquerda.}
	\label{fig::shoulderleft}
\end{figure}

A simulação mostrou que se mantivermos a altura da base em $y=-3220$ mm
(altura mínima) e $x=1200$ mm, são necessárias duas posições em $z$ (ao longo do
trilho) para o revestimento completo da extremidade. É interessante para o
projeto manter altura fixa o quanto possível, pois há redução de grau de
liberdade, e, portanto, redução na complexidade da base mecânica.

Caso haja alteração na altura do robô, por exemplo  $y=-2770$ mm, só será
necessária uma posição em $z$ para a conertura completa da região. Mas é
preferível mover o robô no trilho, em $z$, a mover o robô em altura, em $y$. 

A extremidade superior esquerda necessitou duas posições para a base, mas não
mostrou desafios técnicos.

\paragraph{Extremidade superior direita}

A extremidade superior esquerda possui duas complexidades de revestimento:
posição do ponto devido à altura do robô; e vetor normal, direção de
revestimento. A figura~\ref{fig::shoulderright} mostra a discretização da pá na
extremidade esquerda: pontos azuis são pontos revestidos na tolerância de $30^o$; em preto, pontos revestidos sem tolerância; e em vermelho, pontos que
não foram revestidos para esta posição do robô.

\begin{figure}[!ht]
	\centering	
	\includegraphics[width=.5\columnwidth]{method/figs/shoulderright.png}
	\caption{Estudo de revestimento para a extremidade superior direita.}
	\label{fig::shoulderright}
\end{figure}

A simulação mostra que, mesmo se utilizarmos a altura máxima para a base
$y=-2720$ mm e mantivermos $x=1200$ mm, não há posição em $z$ (ao longo do
trilho) para revestir por completo a extremidade. Aproximando o manipulador da
pá, novos pontos são revestido, mas mesmo em $x=230$ mm o
revestimento não é completo. O mesmo teste foi feito para diferentes
ângulos do rotor, mas os resultados não são favoráveis, pois conforme o rotor
gira, a pá se afasta do robô.

Para $y=-3220$ mm e $x=1200$ mm, nenhum ponto da extremidade superior direita é
revestido, logo outras estratégias devem ser adotadas. A extremidade superior
direita mostrou grande complexidade técnica e não foi possível encontrar uma
solução viável para o 2º trilho a fim de revestí-la.

\paragraph{Conclusão da simulação de extremidades}

Os resultados da simulação das extremidades da pá mostraram que três das
quatro extremidades da pá podem ser revestidas sem problemas técnicos, e
mantendo fixas as variáveis $y=-3220$ mm (referência global) e $x=1200$ mm de
distância em relação à pá. A extremidade superior direita ainda não apresenta
solução de revestimento utilizando o trilho 2 com as tolerâncias especificadas.

A rotação da turbina também foi simulada de $0^o$ a $30^o$. Entretanto, como
em $0^o$ as três extremidades foram revestidas sem alteração de dois graus de
liberdade $x$ e $y$, escolheu-se manter o rotor em $0^o$ para esta aplicação.

\subsubsection{Teste de revestimento completo e novas soluções de base}

Após a avaliação das extremidades da pá, deve-se ainda simular o revestimento
total, pois a superfície da pá é muito irregular, podendo se aproximar ou se
afastar do robô para certas posições de base. Além disso, foram abordadas novas
estratégias para a solução de revestimento da extremidade direita da pá. A
simulação de teste de toda a pá considerou as seguintes variáveis:

\begin{itemize}
  \item O ângulo de ataque das pás da turbina variam de $0^o$ a $29^o$. 
  O acréscimo deste
  grau de liberdade buscou solucionar o problema da extremidade direita.
  \item O rotor da turbina foi girado de $0^o$ a $30^o$ com passo de $3^o$.
  Manteve-se este grau de liberdade, pois em conjunto com o giro da pá, o
  problema da extremidade superior direita poderia ser solucionado.
  \item A distância entre extremidade da pistola e pá pode variar $235
  \pm 5$ mm.
  \item O ângulo entre a pistola e o plano da pá pode variar $90^o \pm
  30^o$. Alguns testes utilizaram a tolerância limite de $60^o$ como tentativa
  de revestir a extremidade direita superior.
  \item A posição em $y$ do robô foi mantida fixa. $-3220$ mm na referência
  global.
  \item A posição em $x$ do robô foi mantida fixa. $1200$ mm de
  distância em relação a pá.
  \item A posição em $z$ do robô foi amostrada uniformemente em 10 pontos ao
  longo da pá. A equipe de mecânica restringiu o movimento em $z$ tal que
  $-1240 < z < 1240$, garantindo para este $y$ mínimo ($-3220$ mm) espaço
  suficiente para a construção do trilho.
\end{itemize}

A restrição da mecânica para a construção do trilho $-1240$ mm $< z <$ $1240$ mm
prejudica o revestimento na lateral direita da pá, região em que o aro câmara
está próximo de $20^o$. A figura~\ref{fig::simcomp1_1} mostra a
discretização completa da pá, nas condições em que o rotor está $0^o$ e a pá
$24^o$:
em preto, pontos revestidos; e em vermelho, pontos que não foram revestidos. As
figuras seguintes adotarão a mesma legenda de cores para os pontos revestidos.
Como pode ser visto, não foi possível revestir os pontos da lateral direita, a extremidade superior direita e a extremidade superior esquerda.

\begin{figure}[!ht]
	\centering	
	\includegraphics[width=.5\columnwidth]{method/figs/simcomp1_1.png}
	\caption{Simulação de revestimento completo da pá, considerando as
	restrições mecânicas da base, ângulo $0^o$ do rotor e $24^o$ da pá.}
	\label{fig::simcomp1_1}
\end{figure}

\paragraph{Extremidade superior esquerda}
Há três possíveis soluções para revestir a extremidade superior esquerda:
elevação da base do robô; aumentar tolerância de ângulo de revestimento para
$60^o$; e rotação da turbina para $-15^o$. Como esta
região da pá tem inclinação projetada para o robô, alterar o ângulo da pá de $24^o$ para $0^o$ não favorece o
revestimento (figura~\ref{fig::simcomp1_4}). Em termos de operação, as três
medidas possuem desvantagens: rotacionar a turbina é complexo, visto que será
necessário realizar um procedimento para transportar os equipamentos a uma área
de segurança e fazer a recalibração; elevar o robô apresenta complexidade
mecânica, já que exige mais um grau de liberdade da base, e fazer recalibração;
aumentar a tolerância do ângulo de revestimento tem complexidade menor, mas
aumenta a perda de material de revestimento.

A figura~\ref{fig::simcomp1_5} mostra a discretização completa da pá, nas
condições em que o rotor está $-15^o$ e a pá $24^o$. Conforme o rotor é
girado no sentido horário (negativo), o revestimento na extremidade superior
esquerda da pá é completa, porém o lado direito fica prejudicado.

\begin{figure}[!ht]
	\centering	
	\includegraphics[width=.5\columnwidth]{method/figs/simcomp1_5.png}
	\caption{Simulação de revestimento completo da pá, considerando as
	restrições mecânicas da base, ângulo $-15^o$ do rotor e $24^o$ da pá.}
	\label{fig::simcomp1_5}
\end{figure}

Quando aumentamos a tolerância de ângulo de revestimento
para $60^o$, obtemos a figura\ref{fig::simcomp1_3}. A figura mostra que foi
possível revestir toda a extremidade superior esquerda, salvo pontos de colisão
com o rotor. Entretanto, é importante a base ter o grau de liberdade em $y$ para
suprir eventuais problemas de modelagem.

\begin{figure}[!ht]
	\centering	
	\includegraphics[width=.5\columnwidth]{method/figs/simcomp1_3.png}
	\caption{Simulação de revestimento completo da pá, considerando as
	restrições mecânicas da base, ângulo $0^o$ do rotor e $24^o$ da pá,
	tolerância de $60^o$ de revestimento.}
	\label{fig::simcomp1_3}
\end{figure}

Ao elevarmos a base $y+500 mm$, obtemos a figura~\ref{fig::simcomp1_6}. A figura
mostra que foi possível revestir toda a extremidade superior esquerda e, ainda,
alguns pontos na lateral direita que não haviam sido revestidos. É muito
provável que este grau de liberdade da base seja projetado, a fim de o projeto
não ficar dependente dos movimentos da turbina.

\begin{figure}[!ht]
	\centering	
	\includegraphics[width=.5\columnwidth]{method/figs/simcomp1_6.png}
	\caption{Simulação de revestimento completo da pá, considerando as
	restrições mecânicas da base, ângulo $0^o$ do rotor e $24^o$ da pá,
	$y+500 mm$.}
	\label{fig::simcomp1_6}
\end{figure}

\paragraph{Lateral direita}
Em relação à lateral direita da pá, a ideia imediata é rotar a turbina,
esperando que o lado direito se aproxime do robô. Outras possibilidades é
girar a pá em seu próprio eixo para $0^o$, em vez de $24^o$. As desvantagens de
cada solução se assemelham às discutidas previamente: rotar a turbina exige
transporte dos equipamentos à área de segurança, remontagem e recalibração;
girar a pá só é possível através do circuito hidráulico e talvez não seja
possível após início da operação.

A figura~\ref{fig::simcomp1_2} mostra a discretização completa da pá, nas
condições em que o rotor está $15^o$ e a pá $24^o$. Conforme o rotor é
girado no sentido anti-horário (positivo), embora o revestimento aumente na
lateral direita da pá, a lateral esquerda e as áreas inferiores começam a ser
prejudicadas. A extremidade superior direita continua sem ser revestida, como
esperado, vide subseção~\ref{superioresquerda}.

\begin{figure}[!ht]
	\centering	
	\includegraphics[width=.5\columnwidth]{method/figs/simcomp1_2.png}
	\caption{Simulação de revestimento completo da pá, considerando as
	restrições mecânicas da base, ângulo $15^o$ do rotor e $24^o$ da pá.}
	\label{fig::simcomp1_2}
\end{figure}

A figura~\ref{fig::simcomp1_4} mostra a discretização completa da pá, nas
condições em que o rotor está $0^o$ e a pá $0^o$. Conforme a pá é girada, o
revestimento aumenta na lateral direita e se mantém na lateral esquerda. Isso
ocorre, pois o robô se mantém longe do aro câmara no lado direito, já que o aro
ainda não está a $20^o$. Esta é a melhor posição para revestimento da pá,
situação aparente ao método empregado pela Rijeza. Entretanto, esta configuração
fornece pequeno espaçamento entre pás, estreitando a passagem do robô e, muito
provavelmente, não é possível alterar esse ângulo frequentemente já que exige
funcionamento do circuito hidráulico.

\begin{figure}[!ht]
	\centering	
	\includegraphics[width=.5\columnwidth]{method/figs/simcomp1_4.png}
	\caption{Simulação de revestimento completo da pá, considerando as
	restrições mecânicas da base, ângulo $0^o$ do rotor e $0^o$ da pá.}
	\label{fig::simcomp1_4}
\end{figure}

\paragraph{Extremidade superior direita}
A extremidade superior direita requer uma nova estratégia, pois todas as
outras falharam até então: rotacionar turbina, girar pá, elevar o robô no trilho
$y+500 mm$. Não há possibilidade, portanto, de realizar o revestimento a partir
do trilho 2 (posicionamento). A solução encontrada até o momento é rotar a
turbina $45^o$, manter a pá em $24^o$ e posicionar o robô entre as pás, no
trilho 1 (transporte). A desvantagem logística de rotar a turbina é comum às
soluções antigas, porém, esta configuração da turbina já estava prevista para
a entrada do robô no lado do distribuidor.

A figura~\ref{fig::simcomp1_7} mostra a discretização completa da pá, nas
condições em que o rotor está $45^o$ e a pá $24^o$. O robô se encontra
com sua base nas posições destacadas em verde. Como podemos observar, nesta
configuração (robô no trilho 1), é possível revestir a extremidade
superior direita. A fim de melhor aproveitamento logístico da
operação, esta operação deve ser realizada antes da entrada completa
do robô no lado do distribuidor.

\begin{figure}[!ht]
	\centering	
	\includegraphics[width=.5\columnwidth]{method/figs/simcomp1_7.png}
	\caption{Simulação de revestimento completo da pá, considerando as
	restrições mecânicas da base, ângulo $45^o$ do rotor e $24^o$ da pá, robô
	entre as pás.}
	\label{fig::simcomp1_7}
\end{figure}

\paragraph{Conclusão da simulação completa}

A simulação completa da pá mostrou diversas estratégias para o revestimento da
pá, considerando as restrições mecânicas do trilho, o ambiente modelado da
turbina, e as diversas variáveis do processo de revestimento.

A partir dos resultados das simulações, podemos concluir que é possível realizar
o revestimento completo da pá, inclusive das áreas de mais difícil acesso, como
as extremidades. O revestimento completo, no entanto, requer uma extensa
logística de operação. Para cada face da pá, o robô deverá executar o
procedimento tanto no trilho de posicionamento (trilho 2), quanto no trilho de
transporte (trilho 1), a fim de revestir a área mais complexa, extremidade
superior direita. 
\subsection{Planejamento de trajetória}

O conhecimento de todas as regiões que podem ser recobertas pelo robô ajudam na
validação do posicionamento do robô. Porém, ainda é necessário descrever o
caminho a ser percorrido pelo efetuador no espaço de trabalho de maneira a
cumprir os requisitos de revestimento e o respectivo caminho percorrido no espaço de juntas.

A definição desse caminho é facilitada pelo conhecimento analítico da
superfície. A partir dessa descrição são definidos, sobre a pá, uma série
de faixas que segmenta a pá, que serão chamadas de
paralelos.

Os paralelos são espaçados de 3 em 3 milímetros para o cumprimento das exigência
de revestimento e são dispostos de forma a não possuirem intersecção entre si. A
união dos paralelos (considerando uma faixa de 3mm entre eles) inclui todos os
pontos da pá, garantindo seu completo recobrimento.

O deslocamento da ferramenta de revestimento  pelo caminho definido por dois
paralelos é feito por meridianos. Isto é, segmentos que
conectam extremidades dos paralelos e não tem função defininda no processo de
metalização. A vávula deve estar fechada e o revestimento interrompido
durante o tempo que o efetuador percorre todo o meridiano. A razão de existência
dele é meramente definir uma maneira de levar o efetuador de um paralelo a
outro.

\subsubsection{Modelagem da superfície}\label{modelagem}

Existem diversas abordagens matemáticas para descrição de superfícies como:
Parametrização Polinomial; Polinômios em três variáveis; Superfícies de
Bézier (\cite{farin2002curves}); Splines e NURBS (\textit{Non-uniform
rational B-spline}); Subdivisão de superfícies (\cite{peters2008subdivision}); Malhas
poligonais.

Todas essas formas de representar uma superfície, com excessão das malhas,
recaem em alguma instância em uma descrição polinomial. Dentre essas a única
descrição que ocorre de maneira implícita é por polinômios em três variáveis,
descrevendo uma variedade algébrica bidimensional, enquanto as demais são
parametrizações da superfície. 

Por simplicidade, fácil manipulação algébrica e implementação, a descrição
puramente polinomial (implícita) foi escolhida como abordagem inicial. De
maneira geral a superfície é descrita como o conjunto solução sobre os
números reais da equação polinomial ($f(x,y,z)=0$) de grau $N$, dito grau da
superfície, em $x$,$y$ e $z$:
\[\sum\limits_{i+j+k \leq N}^{} C_{i,j,k}x^iy^jz^k = 0\]

Os coeficientes $C_{i,j,k}$, então, são aqueles que descrevem da superfície.
Devido a restrição do grau do polinômio, o número de coeficientes é
$\binom{N+3}{3}$. Podendo ser vistos como coordenadas da superfície num espaço
projetivo de dimensão igual ao número de coeficiente,
$\mathbb{P}^{\binom{N+3}{3}}$, em outras palavras a superfície é invariante a
escalamento dos coeficientes.

Experimentalmente foi indentificado que um polinômio de quarto grau é suficiente
para aproximar toda uma região de interesse da pá, onde será feito o revestimento
para uma posição do robô, com erro submilimétrico. Nesse caso, o número de
coeficientes que devem ser identificados é $\binom{7}{3}$, ou seja, 35.

Com base no artigo de \cite{juttler2002least}, a conversão da descrição da
superfície de nuvem de pontos para uma descrição analítica foi feita utilizando
a informação da direção da normal à superfície em cada ponto, ou seja, a superfície analítica deve não apenas passar
próxima aos pontos da nuvem como deve também ter seu vetor normal similiar à
normal desses pontos.

Explorando o fato que polinômio são lineares em seus coeficientes, um sistema
superdeterminado, a ser resolvido por mínimo quadrados(\textit{curve fitting}
\cite{arlinghaus1994practical}) , foi contruído a partir do cálculo dos termos
do polinômio em cada ponto da nuvem (fazendo $f(x,y,z)=0$) e da avaliação da
normal em cada ponto, que deveria concordar com o gradiente do polinômio (ou
seja $\nabla f(x,y,z) = \overrightarrow{n}$, onde $\overrightarrow{n}$ é a
normal no ponto $(x,y,z)$ da nuvem), dessa forma o peso dado aos vetores normais
das amostras é igual ao peso das amostras.

%Elael Modelo Polinomial multivarial -> extrapolado para a pá inteira
% PREMISSAS!!


% 
% Splines
% Bézier Surface
% Runge's phenomenon
% Multivariate Polynomial fitting

\subsubsection{Cálculo dos paralelos}\label{paralelos}
% como subdividir em segmentos as regiões PREMISSAS!!
Na literatura, há diversas formas de dividir a superfície a ser revestida em
subregiões. Em \cite{from2010off}, por exemplo, um manipulador realiza a pintura
de uma superfície (\textit{spray gun}) cobrindo subregiões de um plano, projeção
%TODO nao entendi
 da superfície (figura~\ref{fig::pal}). Outra possibilidade é,
em funções
paramétricas, realizar uma trajetória semelhante à figura~\ref{fig::pal} no
espaço dos parâmetricos, cuja transformação (jacobiano) mapeará nos 'cortes' curvos da
superfície.

\begin{figure}[!ht]
	\centering	
	\includegraphics[width=0.7\columnwidth]{method/figs/planejamento/pal.png}
	\caption{Subregiões de uma superfície.}
	\label{fig::pal}
\end{figure}



A superfície descrita na seção~\ref{modelagem} é uma equação implícita, na forma
$f(x,y,z)=0$. Neste caso, as trajetórias a serem percorridas pelo manipulador
podem ser obtidas através da interseção (cortes) entre planos uniformemente
espaçados e a superfície, o que gerará curvas ao seu longo. Uma
ideia semelhante e propícia devido à geometria do rotor, é gerar as curvas a partir da interseção
entre esferas e a superfície. As figuras~\ref{fig::interfrontal}
e~\ref{fig::interiso} mostram duas visões de duas interseções entre esferas e
a pá, onde as interseções estão representadas em vermelho, e as esferas em
azul claro. Os mesmos cortes podem ser observados entre esferas e o modelo
algébrico da pá, em figura~\ref{fig::intergeo}, na qual a pá está representada
em vermelho, as esferas estão em cinza claro, e as interseções são as curvas
sombreadas em cinza na pá.

A superfície descrita na seção~\ref{modelagem} é um subconjunto de uma
variedade algébrica de dimensão dois, representada na forma $f(x,y,z)=0$ como
uma imersão em $\mathbb{R}$, sendo assim uma superfície implícita onde
$f(x,y,z)$ é um polinômio nas três variáveis .
% Neste caso, trajetórias a serem percorridas pelo manipulador podem ser obtidas através da interseção (cortes)
% entre planos uniformemente espaçados e a superfície, o que gerará curvas ao
% longo da superfície. Uma ideia semelhante e propícia devido à geometria do
% rotor, é gerar as curvas a partir da interseção entre esferas e a superfície. 
As
figuras~\ref{fig::interfrontal} e~\ref{fig::interiso} mostram duas visões de
duas interseções entre esferas e a pá, onde as interseções estão representadas
em vermelho. Os mesmos cortes podem ser observados entre esferas e o modelo
algébrico da pá, em figura~\ref{fig::intergeo}.


\begin{figure}[!ht]
	\centering
	\includegraphics[width=0.7\columnwidth]{method/figs/planejamento/intersecao_frontal.PNG}
	\caption{Interseção esfera-pá, vista frontal.}
	\label{fig::interfrontal}
\end{figure}

\begin{figure}[!ht]
	\centering
	\includegraphics[width=0.6\columnwidth]{method/figs/planejamento/intersecao_iso.PNG}
	\caption{Interseção esfera-pá, vista isométrica.}
	\label{fig::interiso}
\end{figure}

\begin{figure}[!ht]
	\centering
	\includegraphics[width=0.7\columnwidth]{method/figs/planejamento/intersecao_geogebra.png}
	\caption{Interseção esfera-modelo pá.}
	\label{fig::intergeo}
\end{figure}

A interseção de duas superfícies, a superfície da esfera
$g(x,y,z)=x^2+y^2+z^2-R^2=0$ e a superfície da pá $f(x,y,z)=0$, gera o caminho
que deve ser percorrido pelo robô. Porém, resolver algebricamente
$f(x,y,z)=g(x,y,z)$ é muito custoso, haveria a necessidade de
calcular as soluções para os ângulos das juntas do robô posteriormente, e ainda
realizar cáculos de restrição de borda da superfície, já que a função algébrica
encontrada para a pá é contínua e pode ter comportamento estranho fora da
região de interesse. Portanto, foi desenvolvido um método iterativo para a
computação da trajetória do robô de forma que, ao mesmo tempo que o caminho é
criado, os ângulos das juntas são computados, otimizando localmente a variação
dos ângulos das juntas, verificando restrição de ângulo de revestimento, e
bordas.

O método será explicado a partir de um exemplo genérico: suponha a superfície
algébrica da pá em vermelho, o rotor em preto, e a área que pode ser
revestida dada uma base do robô em amarelo, na figura~\ref{fig::vetores_out}.
Nesta etapa de revestimento (robô nesta posição de base), devem ser calculadas
as curvas (trajetórias). É selecionado o ponto central da núvem de
pontos revestidos (em amarelo), e é calculado o ponto mais próximo à superfície,
representado como ponto $B$ da figura~\ref{fig::vetores_out2}.
$\vec{AB}$ é o vetor normal à esfera, igual a $\vec{OB}$ (origem ao ponto B), e $\vec{BC}$ é o vetor normal à superfície
algébrica da pá, calculado como $\nabla{f} = \vec{f_x,f_y,f_z}$. O vetor
tangente $\vec{BD}$, figura~\ref{fig::vetores_in}, pode ser calculado como
produto vetorial entre o vetor normal à superfície da pá com o vetor normal à
esfera, no ponto B: $\vec{BD} = \vec{BC} \times \vec{AB} = \vec{OB} \times
\nabla{f}$. A integral do vetor tangente irá fornecer a trajetória (região
cinza sombreada, ou em vermelho, como na figura~\ref{fig::interiso}), logo o
caminho (ou trajetória) é calculado por:
$$c = \int \vec{OB} \times \nabla{f} dt.$$ 

Em integrações numéricas, deve-se garantir que o novo ponto de cada iteração
pertença à superfície, logo $B' = B + \int_0^t \vec{OB} \times \nabla{f} dt$, onde $t$ é o passo de integração, não é
suficiente, pois deve-se reprojetar o novo ponto $B'$ na superfície da pá. Isso
é feito por uma otimização, enunciada da seguinte forma:
$$min \left \| B-B' \right \|^2$$
$$s.t. f(B')=0$$
E assim garante-se que o novo ponto $B'$ pertence à superfície da pá.

Em cada passo da integração numérica, deve-se computar a solução dos ângulos das
juntas (cinemática inversa) para o revestimento. Caso não haja solução, ou o
ângulo entre avanço do efetuador e normal da pá seja maior que $30^o$, a trajetória está
concluída e a integração é interrompida. Calculam-se os valores dos ângulos das
juntas em cada passo por uma otimização, enunciada da seguinte forma:
$$min -\nabla{f}\cdot x_T$$
$$s.t. \left \| p_T-B \right \|^2=0$$
Onde $x_T$ são as três primeiras linhas da primeira coluna da transformação
homogênea $T_{bB}$ ($b$ é a base do manipulador), pois o vetor de avanço do
efetuador do manipulador é $x=(1,0,0)$, logo a primeira coluna. E $p_T$ são as
três primeiras linhas da quarta coluna da transformação
homogênea $T_{bB'}$, representando a posição do efetuador.

\begin{figure}[!ht]
	\centering
	\includegraphics[width=0.7\columnwidth]{method/figs/planejamento/vetores_out.png}
	\caption{Vetores de interesse na interseção esfera-pá.}
	\label{fig::vetores_out}
\end{figure}

\begin{figure}[!ht]
	\centering
	\includegraphics[width=0.7\columnwidth]{method/figs/planejamento/vetores_out2.png}
	\caption{Vetores de interesse na interseção esfera-pá.}
	\label{fig::vetores_out2}
\end{figure}

\begin{figure}[!ht]
	\centering
	\includegraphics[width=0.7\columnwidth]{method/figs/planejamento/vetores_in.png}
	\caption{Vetores de interesse na interseção esfera-pá.}
	\label{fig::vetores_in}
\end{figure}

A figura~\ref{fig::path_openrave} mostra duas curvas computadas pelo algoritmo
descrito acima. Os caminhos tem espaçamento exagerado, maior que 3 mm, para
facilitar a visualização. A transição entre os paralelos, no entanto, é
executada por outro algoritmo, que calcula os meridianos do planejamento de
trajetória.

\begin{figure}[!ht]
	\centering
	\includegraphics[width=0.7\columnwidth]{method/figs/planejamento/path_openrave.png}
	\caption{Simulação de trajetória no Openrave.}
	\label{fig::path_openrave}
\end{figure}

\subsubsection{Cálculo dos meridianos}
Os paralelos, ou caminhos ``horizontais'', são computados pelo algoritmo
descrito na subseção~\ref{paralelos}. Entretanto, o algoritmo não descreve as
transições entre linhas horizontais, como se o manipulador ``pulasse''
de um paralelo a outro, o que não pode acontecer, já que o caminho deve ser
contínuo. Dessa forma, há a necessidade de computação das curvas de transição,
os caminhos ``verticais'', ou meridianos da superfície da pá.

Ao fim da execução do cálculo de um paralelo (por exemplo, ao fim do
cálculo da curva em vermelho da figura~\ref{fig::interiso}), o
efetuador estará apontando para o último ponto com solução viável neste
paralelo, dentro das restrições de ângulo de revestimento, no lado esquerdo ou direito. A partir
deste ponto extremo (borda), o manipulador deverá ``descer'' ou ``subir'' pelo
meridiano, até encontrar outro paralelo, isto é, encontrar outra curva que
satisfaça $f(x,y,z)=g(x,y,z)$.

O método será explicado a partir de um exemplo genérico: suponha que o efetuador
do manipulador se encontra como na figura~\ref{fig::path_openrave} (borda
direita), isto é, na extremidade direita de um paralelo $c_1$. Caminhar em um
meridiano significa integrar o vetor tangente perpendicular ao encontrado por $\vec{BD} = \vec{OB} \times
\nabla{f}$, logo o caminho pelo meridiano pode ser calculado
como:
$$m_{12} = \int (\vec{OB} \times \nabla{f})\times \nabla{f} dt$$
o que irá gerar um caminho de descida pelo meridiano. Em cada passo de
integração numérica, o novo ponto $B' = B + \int_0^t (\vec{OB} \times
\nabla{f})\times \nabla{f} dt$ deve ser projetado na superfície, como em
na subseção~\ref{paralelos}, pela otimização:
$$min \left \| B-B' \right \|^2$$
$$s.t. f(B')=0$$
Além disso, em cada passo deverá ser verificado se o caminho já alcançou o
próximo paralelo $c_2$, isto é, se o ponto pertence à esfera de raio
$R_2=R_1+0.003$ (em milímetros). Observe que se o o caminho passar do próximo
paralelo, o caminho deve ser feito no sentido contrário com passo menor, isto é:
$$m_{21} = \int -(\vec{OB} \times \nabla{f})\times \nabla{f} dt$$

Na figura~\ref{fig::meridianos}, os meridianos estão representados em verde.

\begin{figure}[!ht]
	\centering
	\includegraphics[width=\columnwidth]{method/figs/planejamento/meridianos.png}
	\caption{Meridianos da pá.}
	\label{fig::meridianos}
\end{figure}

\subsection{Conclusão}

No âmbito do controle e planejamento de trajetória, o método desenvolvido já se
mostrou capaz de analisar a superfície da pá, segmenta-la em regiões e definir
caminhos para serem percorridos, tanto pelo efetuador (no espaço de trabalho)
quanto pelas juntas (no espaço de juntas).

Porém, validações da precisão do resultado e análise de colisão devem ser mais
exploradas. Também devem ser julgadas pequenas modificações do método como
aplicação de minímos quadrados móveis na definição da superfície, técnica 
que possibilitaria um maior controle do erro localmente.






% \subsection{Sistema de controle}
% % superfinal - citando tipos de controle que cobrem os apectos de caminho