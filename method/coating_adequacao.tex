\subsection{Adequação dos equipamentos}
Devido à necessidade de verificar o funcionamento dos equipamentos a partir das
modificações necessárias para realizar o serviço de revestimento no ambiente do
circuito hidráulico da UHE Jirau, foram conduzidos testes que representem a
operação no ambiente proposto.

A preparação da superfície é uma etapa crítica da operação de aspersão térmica
pois o grau de adesão do revestimento ao substrato está diretamente relacionado
à limpeza e rugosidade da peça que deve ser metalizada. O jateamento abrasivo é
realizado seguindo normas de preparação de superfície para garantir o sucesso
na aplicação de revestimentos por aspersão térmica. Para a detecção de falhas no
metal base, uma inspeção prévia à operação de revestimento é necessária, pois
falhas estruturais no material serão copiadas pelo revestimento. Por exemplo,
uma trinca no metal base não pode ser reparada pelo processo aspersão
térmica. Os materiais depositados por essa técnica não adicionam resistência
mecânica ao substrato.

Na primeira etapa da preparação da superfície deve ser feita limpeza de
partículas, óleos e umidade. A umidade na superfície é minimizada através
de pré-aquecimento, garantindo a efetividade do jateamento abrasivo. Ao longo de
todo ciclo de metalização, a limpeza da superfície deve ser preservada, sem
contaminação por partículas, umidade e oleosidade. O Jateamento é realizado a
partir de um jato de ar comprimido contendo partículas abrasivas (Al2O3)
direcionadas à superfície que deve ser limpa. Durante a aspersão, as partículas
semi-fundidas aderem à superfície pelo impacto e moldagem à rugosidade da
superfície e posterior contração causando um ancoramento mecânico. Por este
motivo, a criação de uma superfície rugosa aumenta a aderência e a coesão entre
as partículas devido às tensões superficiais de contração, inter-travamento
entre camadas, aumento da área exposta e descontaminação da superfície.

As condições de jateamento, como tipo de abrasivo, pressão do jato, fatores
ambientais e contaminação da superfície, devem ser atentados no momento do
procedimento. No ambiente da turbina hidrelétrica, deve ser observada a
formação de orvalho na superfície da pá e, se necessário, realizar
preaquecimento. A principal modificação existente no ambiente proposto está
relacionada à perda de pressão do jato, devido ao grande comprimento de
mangueiras utilizados: do reservatório ao bocal.

Testes comparativos de jateamento foram realizados com perda de carga em
altura, comparando-se as rugosidades nos corpos de prova na medida em
que a altura do bico do jato aumenta em relação ao reservatório de óxido. Foram
utilizadas alturas de 0 m; 1.7 m; 3.3 m e 4.4 m. A análise de desempenho é
realizada posteriormente com o rugosímetro. Duas pressões são também avaliadas
para verificar a perda de rendimento na criação da rugosidade, a partir de um
regulador de pressão.

A medição de rugosidade é realizada através de um dispositivo com sistema de
apalpamento através de uma linha na superfície. O principal parâmetro de
medição é a rugosidade Ra que é a média aritmética dos valores das ordenadas de
afastamento dos pontos de perfil de rugosidade dentro do percurso de medição. A
rugosidade ideal para o recebimento do material de coating deve ser criada a
partir do jateamento com grãos angulares de alumina e deve ficar com valor
final Ra mínimo de 3.2 microns.

A Tabela~\ref{tab:hvof_tab1} apresenta os resultados obtidos para 5 e
8 bar de pressão de jateamento em corpos de prova com dureza de 40 HRC.

Houve decréscimo na rugosidade obtida com a redução da pressão de jateamento de
8 para 5 bar. Porém, para uma mesma pressão, o rendimento se manteve constante
com o aumento da altura. Comparando os valores de 0 m com 4.4 m houve inclusive
acréscimo da rugosidade, provavelmente pelo fato de a mangueira ser muito
longa, a altura ajudou a reduzir o número de curvas do sistema.

\begin{table}[]
\centering
\caption{Resultados dos testes de preparação de superfícies}
\label{tab:hvof_tab1}
\begin{tabular}{ccc}
\hline
\begin{tabular}[c]{@{}c@{}}Pressão\\ de Operação (Bar)\end{tabular} & \begin{tabular}[c]{@{}c@{}}Altura de \\ jateamento (m)\end{tabular} & \begin{tabular}[c]{@{}c@{}}Média das Rugosidades \\ Ra Medidas (µm)\end{tabular} \\ \hline
\multicolumn{1}{|c|}{5}                                             & \multicolumn{1}{c|}{0}                                              & \multicolumn{1}{c|}{3.11}                                                        \\ \hline
\multicolumn{1}{|c|}{5}                                             & \multicolumn{1}{c|}{1.7}                                            & \multicolumn{1}{c|}{3.07}                                                        \\ \hline
\multicolumn{1}{|c|}{5}                                             & \multicolumn{1}{c|}{3.3}                                            & \multicolumn{1}{c|}{2.63}                                                        \\ \hline
\multicolumn{1}{|c|}{5}                                             & \multicolumn{1}{c|}{4.4}                                            & \multicolumn{1}{c|}{3.34}                                                        \\ \hline
\multicolumn{1}{|c|}{8}                                             & \multicolumn{1}{c|}{0}                                              & \multicolumn{1}{c|}{3.61}                                                        \\ \hline
\multicolumn{1}{|c|}{8}                                             & \multicolumn{1}{c|}{1.3}                                            & \multicolumn{1}{c|}{4.15}                                                        \\ \hline
\multicolumn{1}{|c|}{8}                                             & \multicolumn{1}{c|}{3.3}                                            & \multicolumn{1}{c|}{4.44}                                                        \\ \hline
\multicolumn{1}{|c|}{8}                                             & \multicolumn{1}{c|}{4.4}                                            & \multicolumn{1}{c|}{4.62}                                                        \\ \hline
\end{tabular}
\end{table}
