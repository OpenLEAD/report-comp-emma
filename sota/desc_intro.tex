O fenômeno de cavitação e abrasão em hidroturbinas provoca desgaste
superficial por erosão e alteração do perfil
hidráulico da pá, gerando redução da eficiência na geração de energia.
Uma solução preventiva é o revestimento por metalização das pás, o qual aumenta a eficiência na
geração de energia por gerar uma estrutura mais lamelar, e fornece maior
resistência a desgastes. No caso da usina hidrelétrica Jirau, o revestimento
das pás é realizado antes da montagem e instalação da turbina, porém devido ao grande número de
partículas e sedimentos que o rio madeira carrega e à cavitação, o revestimento
deve ser aplicado novamente em intervalos curtos de tempo
\citep{santa2009slurry}. A desmontagem da turbina, aplicação de novo
revestimento nas pás e remontagem são um processo muito custoso e deverá ser
feito regularmente. Portanto, há a necessidade de o procedimento ser
executado dentro do aro câmara, \textit{in situ}, onde as pás são instaladas.

A cavitação é a formação de cavidades de vapor (bolhas), em um líquido, devido a
quedas repentinas de pressão. Quando o líquido é sujeito ao aumento de pressão,
as bolhas implodem, ocasionando ondas de choque \citep{brennen2013cavitation}.

Em hidroturbinas, o fenômeno de cavitação é comum próximo às pás ou
na saída da turbina. O líquido apresenta a combinação
de componentes cinético, potencial gra\-vitacional e energia de fluxo. O
componente cinético é em virtude do fluxo da água (velocidade), o potencial tem
relação com a altitude, e a energia de fluxo é energia que um fluido contém
devido à pressão que possui. De acordo com o princípio de Bernoulli, o princípio
da conservação para os fluidos, implica-se que, para uma mesma altitude, o
aumento da componente cinética acarreta em uma diminuição da pressão, ocorrendo
cavitação. 

Quando há cavitação, a formação de bolhas grandes altera as características do
escoamento, ocasionando oscilações ou vibrações na máquina que, por
conseqüência, prejudicam o rendimento do sistema hidráulico. As bolhas
pequenas, ao colapsar, geram ondas de choque de alta frequência, podendo provocar erosões se
próximo à superfície metálica.

Além da cavitação, como a água atravessa o aro câmara em grande velocidade, o
acúmulo de sedimentos irá provocar desgaste abrasivo, isto é, perda de material
pela passagem de párticulas rígidas. 

% Aqui foram acrescentados os problemas do sistema robótico pela visão da UFRJ e
% pela Rijeza.
Os principais problemas encontrados durante o desenvolvimento da solução
são: acesso e transporte do robô para dentro da turbina; movimentação e
posicionamento do robô no ambiente escorregadio, inclinado e curvo; calibração,
e localização do robô e da pá; planejamento de trajetórias; controle do
manipulador; monitoramento do processo, em tempo real; selecionar o equipamento
adequado para realizar a metalização; definir as regiões das pás mais afetadas
pelo desgaste; e selecionar o material adequado para retardar o desgaste da
superfície. Nesta seção, são apresentadas as formas de reduzir os danos da
cavitação pela tecnologia de revestimento por metalização, a contextualização
do problema no caso da usina hidrelétrica Jirau e as tarefas que um sistema
robótico deve realizar para solucionar o problema.

