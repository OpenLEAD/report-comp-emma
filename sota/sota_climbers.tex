\subsection{Robôs escaladores}\label{sota_climbers}
Robôs escaladores são sistemas capazes de sustentar seu próprio peso contra a
gravidade, movendo-se em simples ou complexas estruturas geométricas, como
paredes, tetos e telhados, palhetas de turbinas e plantas nucleares.
Essa classe de robôs oferece eficiência operacional em ambientes
de alta periculosidade, sendo utilizados visando saúde e segurança dos
trabalhadores, como em inspeção e limpeza de arranha-céus, diagnóstico de
tanques de armazenamento em plantas nucleares, solda e manutenção de cascos de
navios e palhetas de turbinas \citep{armada2003application}. 

Os grandes desafios nos projetos de sistemas escaladores são mobilidade e
aderência, além de consumo de energia, capacidade de carga e peso. Em
\cite{modular} e \cite{climbsurv}, os robôs escaladores são divididos em tipos
de locomoção:
pernas; como andador; utilizando segmentos deslizantes; rodas; esteiras; avanço
pendurado por braços; por cabos; e biomimética. E categorias de adesão: sucção
ou pneumática; magnética; eletrostática; química; preensão; e híbrida.

No caso específico deste estudo da arte, destacam-se os robôs escaladores com as
seguintes aplicações:

\begin{itemize}
  \item \emph{Navios e turbinas}: RRX3 para soldagem
  \citep{rrx3}, \emph{Climbing Robot for Grit Blasting} para limpeza
  \citep{crgb} e ICM Robot para inspeção \citep{icm};
  \item \emph{Industrial}: ROMA II \citep{roma} e
  CROMSCI \citep{CROMSCI}, ambos para inspeção; 
 \item \emph{Planta petroquímica}: TRIPILLAR \citep{tripillar} para inspeção.  
\end{itemize}

O RRX3 (figura~\ref{rrx3}), Daewoo Shipbuilding and Marine Engineering, é um
robô para a soldagem de casco de navios. Possui adesão por preensão, locomoção transversal utilizando segmentos deslizantes e locomoção
longitudinal por rodas. Possui um manipulador de 1.5 m com três juntas
prismáticas e três juntas de revolução (3P3R) para a operação de soldagem. 

As características principais do robô são: base e manipulador com
capacidades de carga de 120 kg e 5 kg, respectivamente; manipulador com precisão
milimétrica e efetuador de baixa velocidade; robustez para operar em ambiente de
alta periculosidade; opera instrumento de solda; e locomoção transversal é
restrita à aplicação.

\begin{figure}[ht]
\centering
\includegraphics[width=\columnwidth]{sota/figs/climbers/RRX3_moving.jpg}
\caption{Translação horizontal do robô RRX3.}
\label{rrx3}
\end{figure}

O \emph{Climbing robot for Grit Blasting} (figura~\ref{grit}), University of
Coruna, é um robô para jateamento abrasivo em navios. O robô utiliza duas plataformas deslizantes com sistema de adesão por
ímã magnético. Os módulos apresentam movimentação relativa entre si e pode rotar
para compensar as curvaturas do casco do navio ou desviar de objetos. 

As características principais do robô são: base com
capacidade de carga de sistema abrasivo semelhante a HVOF; base com
locomoção de precisão milimétrica; locomoção ampla, mas não aplicável a
estruturas complexas; e não possui manipulador, sendo necessário percorrer todo
o casco.

\begin{figure}[ht]
\centering
\includegraphics[width=\columnwidth]{sota/figs/climbers/grit.png}
\caption{Climbing robot for Grit Blasting}
\label{grit}
\end{figure}

\emph{The Climber} (figura~\ref{icm}), ICM Robotics, é um robô para inspeção de
turbinas eólicas, remoção de revestimento, limpeza de superfície, e aplicação de revestimento.
Possui adesão pneumática (sucção) e locomoção por esteiras. 

As características principais do robô são: base com capacidade de carga de 25
kg; base com locomoção de precisão milimétrica; manipulador modular pode ser
acoplado à base; manipulador de dimensão reduzida e baixa velocidade; e
locomoção apresenta restrição a algumas curvaturas acentuadas.

\begin{figure}[ht]
\centering
\includegraphics[width=\columnwidth]{sota/figs/climbers/icm.png}
\caption{Robô The Climber da ICM Robotics}
\label{icm}
\end{figure}

O ROMA II (figura~\ref{roma2}), Universidade Carlos II de Madrid, é um robô para
inspeção de ambientes complexos. A sua tecnologia de adesão é pneumática (sucção) e
locomove-se como uma lagarta (biomimética). Sua movimentação e planejamento de
trajetória são realizados de maneira ótima de forma a garantir estabilidade e
evitar obstáculos. 

As características principais do robô são: base com grande capacidade de carga;
base com locomoção de precisão milimétrica; não possui manipulador; locomoção em
ambientes de grande complexidade.

\begin{figure}[ht]
\centering
%\includegraphics[width=8.4cm]{sota/figs/climbers/roma2.jpg}
\includegraphics[width=\columnwidth]{sota/figs/climbers/roma2.jpg}
\caption{ROMA II.}
\label{roma2}
\end{figure}

CROMSCI (figura~\ref{cromsci}), Kaiserslautern University of Technology, é um
robô autônomo para inspeção de grandes paredes de concreto, como pilares de pontes, barragens. Seu
sistema de adesão é composto por sete câmaras de vácuo (sucção), com um sistema
de controle por válvulas e sensores de pressão para reagir rapidamaente a
condições adversas. Locomove-se com rodas omnidirecionais para locomoção.

As características principais do robô são: base com pouca capacidade de
carga; base com locomoção de precisão milimétrica; não possui manipulador; e
apresenta baixa velocidade.

\begin{figure}[ht]
\centering
\includegraphics[width=\columnwidth]{sota/figs/climbers/cromsci.jpg}
\caption{Robô CROMSCI.}
\label{cromsci}
\end{figure}

TRIPILLAR (figura~\ref{tripillar}), École polytechnique fédérale de Lausanne, é
um robô escalador de pequeno porte (96 x 46 x 64 mm) desenvolvido para a inspeção de plantas
petroquímicas. Utiliza um sistema como pernas de lagarta magnéticas em um
formato triangular. Locomove-se por esteiras.

As características principais do robô são: base com pouca capacidade de
carga; base com locomoção de precisão milimétrica; sistema robusto a aplicações
em ambientes de alta periculosidade; sistema de controle simples; robô de
pequenas dimensões; não possui manipulador; sistema ainda não testado em estruturas geométricas complexas.


\begin{figure}[ht]
\centering
\includegraphics[width=\columnwidth]{sota/figs/climbers/tripillar.png}
\caption{Robô TRIPILLAR.}
\label{tripillar}
\end{figure}
   
Os robôs escaladores são utilizados em diversas aplicações e possuem diferentes
soluções de aderência e locomoção, como foi exposto nesta subseção. Não há,
até o momento, um robô escalador que possui todas as características
exigidas para a tarefa de HVOF em pás de turbinas, porém a adaptação de
alguns desses sistemas, como \emph{The Climber} da ICM Robotic, pode gerar
soluções completas.

As vantagens e desvantagens para solução de robôs escaladores são:

\textbf{Vantagens:}
\begin{itemize}
  \item Facilidade de instalação;
  \item Manipulador de pequenas dimensões, já que o robô se movimenta sob a pá
  da turbina;
  \item Base de pequenas dimensões;
  \item Pouco peso;
  \item Autonomia durante a operação em uma pá; 
\end{itemize}

\textbf{Desvantagens:}
\begin{itemize}
  \item Sistema de locomoção complexo com desvio de obstáculo e planejamento de
  trajetória;
  \item Desafio mecânico na construção de uma estrutura capaz de sustentar seu
  peso mais o manipulador com sistema HVOF;
  \item Robô deve ser manualmente instalado em cada pá ou um complexo sistema
  de locomoção por braços deverá ser desenvolvido;
  \item Sistema de segurança do robô deverá ser bem desenvolvido;
  \item Bateria limitada ou sistema de gerenciamento de umbilical para robôs
  móveis;
\end{itemize}