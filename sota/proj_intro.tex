O projeto de robôs autônomos para HVOF em pás de turbinas hidráulicas contempla
as soluções que atendem a \textbf{todos} os requisitos da aplicação. Dessa
forma, serão idealizados sistemas robóticos com a fusão das tecnologias expostas
na seção~\ref{sota} e no contexto da usina hidrelétrica de Jirau. Também serão
desenvolvidos conceitos para movimentação e adequação dos equipamentos para
HVOF, como pistola, aspirador para recolhimento abrasivo, compressor e outros
elementos da solução.

Na seção~\ref{sec::consideracoes}, os acessos ao aro câmara foram
descritos e suas restrições são fundamentais para a elaboração da solução.
Esta seção é dividida em soluções de sistemas robóticos para os dois tipos
de acessos, já que estes são o fator que mais restringe o desenvolvimento do
sistema robótico por limitar suas dimensões, funcionalidades, e exigir a
idealização conjunta de uma logística de acesso e movimentação do robô pelo aro
câmara.
