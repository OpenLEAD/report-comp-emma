\subsection{Acesso pela jusante}
Como última opção de acesso ao rotor, existe a possibilidade de utilização do
tubo de sucção ou descarga como meio de entrada à turbina. Com o fluxo de água
parado, é possível utilizar o Rio como meio de lançamento do sistema. A
complexidade da operação para utilizar esse acesso é maior, entretanto existem
vantagens que podem tornar essa solução viável e mais atrativa.

\textbf{Vantagens}
\begin{itemize}
  \item Virtualmente nenhuma restrição de tamanho
  \item Flexibilidade de soluções
  \item Facilidade de utilização de um manipulador industrial padrão
  \item Possibilidade de implementação em outras usinas
\end{itemize}

\textbf{Desvantagens}
\begin{itemize}
  \item Complexidade de lançamento e recuperação
  \item Custo
  \item Possibilidade de correnteza
  \item Complexidade logística de transporte entre a entrada do tubo de sucção e
  o aro câmara
  \item Complexidade de prototipação
\end{itemize}

As soluções foram divididas em etapas necessárias para a operação, ou
seja, lançamento e recuperação do sistema, logística de transporte e o robô de
metalização propriamente dito.

Para esse acesso, o maior obstáculo presente é o desenvolvimento de um sistema
de lançamento e recuperação do robô, a partir do rio, até o interior da turbina.
Essa operação deverá ser realizada com a turbina alagada e, em seguida,
pode-se dar início ao processo de drenagem da
mesma.
É importante que o sistema de lançamento seja robusto e garanta o perfeito posicionamento do robô dentro da turbina, assim
como, a sua recuperação, uma vez que erros nesse processo podem significar a
perda completa do sistema.

Primeiramente, a solução deve ser a prova d'água com classificação para pelo
menos 50m de profundidade.
Sendo assim, um vaso de pressão para o transporte do robô até o interior da turbina deve ser
desenvolvido, não havendo necessidade do maniupulador responsável pela
metalização ser a prova d'água. O \textit{container} de transporte
submarino deve ser menor que o tamanho do vão do stoplog, visto que o acesso
mais próximo ao tubo de descarga, pelo rio, se dá por esse vão.
Por outro lado, suas dimensões devem ter um tamanho mínimo que possibilite o
encapsulamento do robô e todo o material de suporte necessário. Há aindaa
necessidade de uma escotilha de acesso de tamanho suficiente para que todo o
sistema seja retirado do interior do \textit{container} submarino.

Para o sistema de lançamento, foi deslumbrada uma estrutura de transporte que
utilizará o pórtico rolante e o trilho guia dos stoplogs. Após a submersão da
estrutura, um mecanismo de lançamento, inspirado em um paletizador, é
responsável pelo posicionamento do \textit{container}, sempre no mesmo ponto em
relação ao tubo de descarga. Com o vaso de pressão posicionado, a turbina deve
ser, então, drenada. Em seguida, o robô pode ser retirado de seu
envólucro e a operação de metalização pode ter seu início. Uma etapa crítica da
operação é a recuperação do sistema, na qual a turbina deve ser novamente
alagada e os os stoplogs retirados. A estrutura de transporte deve, então,
recuperar o \textit{container} transportador na mesma posição em que o sistema
foi lançado. O sucesso dessa operação tem como \textbf{hipótese que a velocidade
de drenagem e a correnteza gerada por essa operação não são suficientes para
retirar o container (mais pesado que a água) de sua posição inicial}. 

A movimentação do robô do ponto de lançamento até o aro câmara deverá ser
realizada a partir da utilização de cordas, roldanas e talhas. Caso necessário,
pode ser desenvolvido um sistema de locomoção com trilhos e/ou rodas atuadas
para o posicionamento automático do robô e, até mesmo, um plano elevado para
transporte.

O robô de metalização pode ter diversos fomatos, mas devido a possilidade de se
utilizar um manipulador industrial padrão, o projeto inicial consiste em uma
base de apoio e um manipulador com alcance para o processamento de uma face da
pá posicionado de frente para pá, ou um manipulador posicionado entre duas pás com
alcance para processar as duas as faces das pás voltadas para ele.












