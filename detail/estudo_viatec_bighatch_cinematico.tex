\subsection{Espaço de trabalho e cinemática do
manipulador}\label{sec::cinematica} 
% Author: Renan

Como já mencionado, o ambiente de simulação
foi desenvolvido utilizando a arquitetura de planejamento Openrave. Para cada manipulador selecionado após a
pesquisa de mercado, serão analisados os reais espaços de trabalho, e o processo de
revestimento da pá em um ambiente simulado que representa as principais
caracterísiticas do ambiente real.

Para gerar o espaço de trabalho, o Openrave utiliza um método de força bruta,
onde são executadas iterações sob iterações de todas as juntas, por seus ângulos
limites e com o passo de ângulo dependendo da resolução do manipulador. O grau
de manipulabilidade do robô é representado por um gradiente de cores, cujo grau
varia do azul claro (menor manipulabilidade) ao vermelho escuro (maior manipulabilidade).
Entende-se por manipulabilidade a capacidade que o robô possui de manipular
objetos em direções específicas, isto é, para uma posição específica é
possível alcançar variadas orientações. Em todas as simulações, a pistola foi
representada como um cilindro de comprimento 300 mm e raio 50 mm, e o efetuador está no extremo do cilindro.

A superfície da pá é amostrada, formando uma grade de tamanho fixo. A técnica
\textit{axis-aligned bounding box (AABB)} é utilizada para obter os
pontos e suas respectivas normais, na superfície da pá. Nesta técnica, a
superfície alvo é inscrita em um bloco, que é uniformemente amostrado. É, então,
realizada uma verificação de colisão entre os pontos amostrados no bloco e a
superfície alvo e, caso haja interseção, o ponto é armazenado junto com sua
normal à superfície. Dessa forma, podemos amostrar a pá e deslocar estes pontos
230 mm em relação à sua normal com a superfície, garantindo a requerimento do
revestimento. A representação dos pontos amostrados e deslocados em relação às
normais da pá podem estão nas figuras~\ref{fig::amostrapa1} e ~\ref{fig::amostrapa2}. 

Utilizando as informações dos pontos amostrados e o espaço de trabalho do
manipulador, foram gerados scripts para calcular a
melhor distância do manipulador em relação a pá, de forma que o maior número de
pontos revestidos com angulação de $90^o$ fossem cobertos.
Essa distância é calculada em relação à normal da pá. Com esse dado, é possível estimar quantas posições da base do
manipulador serão necessários para o revestimento de toda a pá.

\begin{figure}[h!]	
	\centering
	\includegraphics[width=0.6\columnwidth]{detail/figs/bighatch/amostrapa1.png}
	\caption{Pontos amostrados da pá - vista lateral}
	\label{fig::amostrapa1}
\end{figure}

\begin{figure}[h!]	
	\centering
	\includegraphics[width=0.6\columnwidth]{detail/figs/bighatch/amostrapa2.png}
	\caption{Pontos amostrados da pá - vista frontal}
	\label{fig::amostrapa2}
\end{figure}

As tabelas~\ref{tab::robocarac} e ~\ref{tab::robocarac} abaixo resume as
caracterísitcas de cada robô e o estudo cinemático realizado, respectivamente:

\begin{center}
\begin{tabular}{  c | c | c | c  }
  \hline
  \textbf{Robô} & \textbf{Payload (Kg)} & \textbf{Massa (Kg)} & \textbf{Alcance
  (mm)} \\ \hline 
  KR10 & 10 & 56 & 1100 \\ \hline
  MH12 & 20 & 130 & 2551  \\ \hline
  LBR 14 & 14 & 30 & 820 \\ \hline
  SIA20D & 20 & 120 &  910 \\
  \hline
\end{tabular}
\captionof{table}{Caracterísitcas principais dos robôs.}
\label{tab::robocarac}
\end{center}

\begin{center}
\begin{tabular}{  c | c | c }
  \hline
  \textbf{Robô} & \textbf{Pontos revestidos (\%)} & \textbf{Posições de base} \\ \hline 
  KR10 & 24.33 & 13\\ \hline 
  MH12 & 53.3 & 4   \\ \hline
  LBR 14 $\uparrow$ & 17.2 & 13 \\ \hline
  LBR 14 $\rightarrow$ & 17.37 & 13 \\ \hline
  SIA20D $\uparrow$ & 23.14 & 9 \\ \hline
  SIA20D $\rightarrow$ & 24.76 & 9 \\ 
  \hline
\end{tabular}
\captionof{table}{Resumo do estudo
cinemático.}
\label{tab::robocin}
\end{center}

\paragraph{KR 10 R1100 sixx WP (Kuka)}
A figura~\ref{fig::kr10cin1} e figura~\ref{fig::kr10cin2} mostram as vistas
lateral e superior do espaço de trabalho do manipulador, respectivamente. Em
vermelho, estão representados os pontos a serem revestidos e em preto os pontos
que o manipulador foi capaz de revestir.

O script que calcula a melhor posição da base em relaçao à pá retornou a posição
870 mm, sendo que 3825 pontos foram revestidos, representando 24.33\% de toda a
pá. Estima-se que serão necessários, pelo menos, 13 posições para o recobrimento
de toda a pá, figura~\ref{fig::kr10bestpos}.



\begin{figure}[h!]	
	\centering
	\includegraphics[width=\columnwidth]{detail/figs/bighatch/kr10_front.png}
	\caption{Espaço de trabalho do manipulador Kuka KR10 - vista lateral}
	\label{fig::kr10cin1}
\end{figure}

\begin{figure}[h!]	
	\centering
	\includegraphics[width=\columnwidth]{detail/figs/bighatch/kr10_top.png}
	\caption{Espaço de trabalho do manipulador Kuka KR10 - vista superior}
	\label{fig::kr10cin2}
\end{figure}

\begin{figure}[h!]	
	\centering
	\includegraphics[width=0.7\columnwidth]{detail/figs/bighatch/kr10_bestpos.png}
	\caption{Melhor posição para o revestimento - robô KR10 da Kuka.}
	\label{fig::kr10bestpos}
\end{figure}


\paragraph{MH12 (Motoman)}
A figura~\ref{fig::mh12cin1} e figura~\ref{fig::mh12cin2} mostram as vistas
lateral e superior do espaço de trabalho do manipulador, respectivamente.

O script que calcula a melhor posição da base em relaçao à pá retornou a posição
950 mm, sendo que 8379 pontos foram revestidos, representando 53.30\% de toda a
pá. Estima-se que serão necessários, pelo menos, 4 posições para o recobrimento
de toda a pá, figura~\ref{fig::mh12bestpos}.

\begin{figure}[h!]	
	\centering
	\includegraphics[width=\columnwidth]{detail/figs/bighatch/mh12_front.png}
	\caption{Espaço de trabalho do manipulador MH12 - vista lateral}
	\label{fig::mh12cin1}
\end{figure}

\begin{figure}[h!]	
	\centering
	\includegraphics[width=\columnwidth]{detail/figs/bighatch/mh12_top.png}
	\caption{Espaço de trabalho do manipulador MH12 - vista superior}
	\label{fig::mh12cin2}
\end{figure}

\begin{figure}[h!]	
	\centering
	\includegraphics[width=0.8\columnwidth]{detail/figs/bighatch/mh12_bestpos.png}
	\caption{Melhor posição para o revestimento - robô MH12 da Motoman.}
	\label{fig::mh12bestpos}
\end{figure}


\paragraph{LBR iiwa 14 R820 (Kuka)}
O manipulador LBR iiwa 14 R820 possui 7 graus de liberdade e, devido a sua
grande flexibilidade e facilidade de montagem, foram estudadas duas
configurações para a base.

O script que calcula a melhor posição da base na posição vertical em relaçao à
pá retornou a posição 1.06, sendo que 2648 pontos foram revestidos,
representando 17.20\% de toda a pá. Estima-se que serão necessários, pelo menos,
13 posições para o recobrimento de toda a pá, figura~\ref{fig::lbrbestposv}.

\begin{figure}[h!]	
	\centering
	\includegraphics[width=0.7\columnwidth]{detail/figs/bighatch/lbr_bestposv.png}
	\caption{Melhor posição para o revestimento - robô LBR da Kuka com base na
	posição vertical.}
	\label{fig::lbrbestposv}
\end{figure}

O script que calcula a melhor posição da base na posição vertical em relaçao à
pá retornou a posição 1400 mm, sendo que 2730 pontos foram revestidos,
representando 17.37\% de toda a pá. Estima-se que serão necessários, pelo menos,
13 posições para o recobrimento de toda a pá, figura~\ref{fig::lbrbestposh}.

\begin{figure}[h!]	
	\centering
	\includegraphics[width=0.7\columnwidth]{detail/figs/bighatch/lbr_bestposh.png}
	\caption{Melhor posição para o revestimento - robô LBR da Kuka com base na
	posição horizontal.}
	\label{fig::lbrbestposh}
\end{figure}

\paragraph{SIA20D (Motoman)}
O manipulador SIA20D também possui 7 graus de liberdade e, devido a sua
grande flexibilidade e facilidade de montagem, foram estudadas duas
configurações para a base.

O script que calcula a melhor posição da base na posição vertical em relaçao à
pá retornou a posição 1100 mm, sendo que 3638 pontos foram revestidos,
representando 23.14\% de toda a pá. Estima-se que serão necessários, pelo menos,
9 posições para o recobrimento de toda a pá, figura~\ref{fig::sia20dbestposv}.

\begin{figure}[h!]	
	\centering
	\includegraphics[width=0.7\columnwidth]{detail/figs/bighatch/sia20d_bestposv.png}
	\caption{Melhor posição para o revestimento - robô SIA20D da Motoman com base
	na posição vertical.}
	\label{fig::sia20dbestposv}
\end{figure}

O script que calcula a melhor posição da base na posição horizontal em relaçao à
pá retornou a posição 1510 mm, sendo que 3892 pontos foram revestidos,
representando 24.76\% de toda a pá. Estima-se que serão necessários, pelo menos,
9 posições para o recobrimento de toda a pá, figura~\ref{fig::sia20dbestposh}.

\begin{figure}[h!]	
	\centering
	\includegraphics[width=0.7\columnwidth]{detail/figs/bighatch/sia20d_bestposh.png}
	\caption{Melhor posição para o revestimento - robô SIA20D da Motoman com base
	na posição horizontal.}
	\label{fig::sia20dbestposh}
\end{figure}

\paragraph{Tolerância no ângulo de revestimento}
As análises de revestimento dos manipuladores exigiram que a pistola
estivesse com as mesmas direções e sentidos opostos às normais dos
pontos a serem revestidos, isto é, a orientação da pistola é sempre
perpendicular ao plano da pá. Entretanto, pode-se assumir uma tolerância de
$90^o \pm 60^o$ entre a pistola e o plano perpendicular, que foi considerada
na análise puramente geométrica. Como o manipulador MH12 (Motoman) possui
recobrimento de quase todo o alcance vertical da pá (revestimento de cima a
baixo), mostra-se interressante a análise de tolerância neste manipulador, de
forma que haja simplificação das possíveis soluções de bases.

Primeiramente, são armazenados os pontos que o robô não foi capaz de
revestir (pontos em vermelho, nas figuras de espaço de trabalho dos
manipuladores) e suas respectivas normais aos planos tangentes à superfície da
pá. Os pontos são deslocados 230 mm na mesma direção e sentido oposto às suas
respectivas normais, de forma que pertençam à superfície da pá,
ponto $D$ é deslocado até ponto $C$ na figura~\ref{fig::tolerancia1}. Para cada
ponto não revestido, são gerados dois vetores unitários $\overrightarrow{v}$ e
$\overrightarrow{w}$ ortogonais entre si e ao vetor normal $\overrightarrow{N}$,
no plano tangente à superfície da pá, conforme
ilustrado na figura~\ref{fig::tolerancia2}. O vetor $\overrightarrow{N}$ é
girado pelo ângulo de tolerância de revestimento $\theta$ (entre $0^o$ e $60^o$)
em relação ao vetor $\overrightarrow{w}$, gerando o vetor
$\overrightarrow{P_1}$, figura~\ref{fig::tolerancia3}.

\begin{figure}[h!]	
	\centering
	\includegraphics[width=0.7\columnwidth]{detail/figs/bighatch/tolerancia1.png}
	\caption{Ponto D não revestido, deslocado 230 mm da superfície da pá.}
	\label{fig::tolerancia1}
\end{figure}

\begin{figure}[h!]	
	\centering
	\includegraphics[width=0.7\columnwidth]{detail/figs/bighatch/tolerancia2.png}
	\caption{Vetores v e w ortogonais ao vetor normal N.}
	\label{fig::tolerancia2}
\end{figure}

Finalmente, o vetor $\overrightarrow{P_1}$ pode ser girado em relação a
$\overrightarrow{N}$ e todos os vetores que pertencem à tolerância de
revestimento $\theta$ saem do ponto $C$ até um ponto do círculo $h$, como o
vetor exemplo $\overrightarrow{P_2}$, na figura~\ref{fig::tolerancia4}. Observe
que este círculo deve ser discretizado, e cada ponto pertencente ao círculo e sua
respectiva normal (vetor de origem $C$ ao ponto do círculo) devem ser
reavaliados, isto é, verifica-se se o robô alcança o ponto pertencente a $h$ com
pistola de revestivemto apontada a sua respectiva normal.
Se algum ponto do círculo puder ser revestido, o ponto $D$ pode ser considerado
como revestido. No exemplo da figura~\ref{fig::tolerancia4}, o círculo $h$ foi
discretizado em dois pontos $G$ e $H$, e suas normais são $\overrightarrow{P_1}$
e $\overrightarrow{P_2}$, respectivamente. 

\begin{figure}[h!]	
	\centering
	\includegraphics[width=0.8\columnwidth]{detail/figs/bighatch/tolerancia3.png}
	\caption{Vetor N girado pelo ângulo de tolerância de revestimento.}
	\label{fig::tolerancia3}
\end{figure}

\begin{figure}[h!]	
	\centering
	\includegraphics[width=0.8\columnwidth]{detail/figs/bighatch/tolerancia4.png}
	\caption{Circulo $h$ representa todos os pontos equivalentes ao ponto $D$ com
	ângulo de tolerância de revestimento $\theta$.}
	\label{fig::tolerancia4}
\end{figure}

Foram realizadas análises de tolerância para dois robôs: MH12, que apresentou o
maior número de pontos revestido na pá, e LBR R820, que é a única solução viável
de manipulador industrial para o acesso superior. Os ângulos de tolerância foram
variados em $10^o$ e $30^o$. 
